\documentclass{article}

\usepackage[english,italian]{babel} %lingue utilizzabili
\usepackage{graphicx}

\usepackage[utf8]{inputenc}

\usepackage{parcolumns} %testo affiancato in colonna

\usepackage{listings} %Per inserire codice
\usepackage[usenames]{color} %Per permettere la colorazione dei caratteri 

\definecolor{editorGray}{rgb}{0.95, 0.95, 0.95}
\definecolor{editorOrange}{rgb}{1, 0.5, 0} % #FF7F00 -> rgb(239, 169, 0)
\definecolor{editorGreen}{rgb}{0, 0.6, 0} % #007C00 -> rgb(0, 124, 0)

\usepackage[a4paper,top=2cm,bottom=2cm,left=3cm,right=3cm] {geometry} %imposta pagina formato A4 e setta margini a piacimento
\usepackage[pdftex]{lscape} %per settare pagine in landscape


%personalizzazione classe per il codice SQL
\lstnewenvironment{sql}[1][] 
{\lstset{basicstyle=\scriptstyle \ttfamily, columns=fullflexible, keywordstyle=\color{blue}\bfseries, ndkeywordstyle=\color{blue}\bfseries , ndkeywords={references}, numberstyle=\color{red}, commentstyle=\color{editorGreen}, showstringspaces=false, stringstyle=\color{editorOrange},
language=SQL, basicstyle=\small,
numbers=left, numberstyle=\tiny,
tabsize=2, stepnumber=10, numbersep=5pt, breaklines=true, frame=single, rulecolor=\color{black}, #1}}
{\lstset {language=SQL,morekeywords={INT,CHAR,varchar,smallint,numeric}}}{}


%personalizzazione classe per il codice HTML
\lstnewenvironment{html}[1][] 
{\lstset{basicstyle=\scriptstyle \ttfamily, columns=fullflexible,
keywordstyle=\color{blue}\bfseries, ndkeywords={content,=,charset=, id=, width=, height=},	 ndkeywordstyle=\color{editorGreen}\bfseries , numberstyle=\color{red} commentstyle=\color{red}, showstringspaces=false, stringstyle=\color{editorOrange},
language=HTML, basicstyle=\small,
numbers=left, numberstyle=\tiny,
tabsize=2, stepnumber=10, numbersep=5pt, breaklines=true, frame=single, rulecolor=\color{black}, #1}}{}

%personalizzazione classe per il codice PHP
\lstnewenvironment{php}[1][] 
{\lstset{basicstyle=\scriptstyle \ttfamily, columns=fullflexible,
keywordstyle=\color{blue}\bfseries, ndkeywords={content,=,charset=, id=, width=, height=},	 ndkeywordstyle=\color{editorGreen}\bfseries , numberstyle=\color{red} commentstyle=\color{red}, showstringspaces=false, stringstyle=\color{editorGreen},
language=PHP, basicstyle=\small,
numbers=left, numberstyle=\tiny,
tabsize=2, stepnumber=10, numbersep=5pt, breaklines=true, frame=single, rulecolor=\color{black}, #1}}{}

\begin{document}
\title{\textbf{Relazione sul progetto di Basi di dati}}

\author{Giovanni Mazzocchin - 1071619\\ Giacomo Zecchin - 1070122}
\maketitle

\tableofcontents  %CREA INDICE

\newpage


\begin{center}
\textbf{Abstract}
\end{center}
\begin{flushleft}
Il progetto modella il database di una scuola di musica privata.\\
In particolare si vogliono gestire la prenotazione delle lezioni e dei saggi, i dati riguardanti le persone coinvolte (insegnanti,allievi e segretari), l'interno della scuola in cui si sono svolte lezioni o tenuto saggi e gli strumenti in essi contenuti.
Le operazioni tipiche sono dunque la memorizzazione dei dati delle persone (allievi e insegnanti) e la prenotazione delle lezioni e dei saggi a cui esse partecipano.
\end{flushleft}

\selectlanguage{italian}


\section{Descrizione dei requisiti}
Il progetto consiste nella realizzazione di una base di dati che modelli alcune
classi coinvolte nella gestione di una scuola di musica privata.\\
Gli insegnanti, i segretari e gli utenti esterni interagiscono con la base di dati tramite 
un'interfaccia web.\\
È stata realizzata un' interfaccia web privata per insegnanti e segretari attraverso la quale essi possono apportare modifiche al database.\\
Allievi e utenti esterni invece possono visualizzare, tramite il sito web, solo le informazioni pubbliche come gli orari delle lezioni, dei saggi e gli strumenti insegnati nella scuola.\smallskip \\
Si vogliono quindi modellare le entità Persona, Insegnante, Segretario, Allievo, Lezione, Saggio, Interno (che possono essere aule o salette concerti), Strumenti (i quali possono essere stabili o portatili) e CopiaStrumentiScuola .
Le lezioni sono unicamente identificate dal loro Id.\\Di una lezione ci interessano: l'insegnante e l'allievo coinvolti, l'ora di inizio e fine, in che aula si svolge e da quale segretario è stata prenotata.\\
I saggi vengono unicamente identificati con la data/ora di inizio in quanto si svolge al massimo un saggio al giorno.\\
Essi devono obbligatoriamente iniziare dopo le ore 19:30.\\
All'interno del database è lo  presente lo storico di tutti i saggi che sono stati fatti all'interno della scuola quindi, se un insegnate si dimentica di inserirlo, può benissimo rimediare anche in un giorno successivo quello in cui è realmente stato fatto.\\
Di un saggio ci interessano l'insegnante che lo dirige e l'interno della scuola nel quale è tenuto.
Di tutte le persone interessano codice fiscale, nome, cognome e numero di telefono.
Inoltre insegnanti e segretari possiedono una password per accedere alle rispettive aree private.
I segretari possono: aggiungere un allievo al database, aggiungere un insegnante e prenotare una lezione.
Un insegnante può: eliminare un allievo appartenente alla propria classe e prenotare un saggio.
Di un interno (aula o saletta concerti) interessano l'edificio, il piano e il codice della stanza, il numero di strumenti attualmente in esso presenti e il numero massimo di strumenti che possono ospitare.
L'entità Strumenti contiente i vari strumenti della scuola rappresentati a livello concettuale e per ognuno di essi ci interessano: il nome, il tipo, la quantità presente all'interno della scuola , il tecnico che si occupa della riparazione ed il numero di telefono di tale tecnico.
Infine l'entità copie strumenti scuola contiene i dati relativi ad ogni esemplare fisico di strumento all'interno della scuola.
Di ogni copia di strumento ci interessano: l'id della copia, che lo identifica univocamente, il nome dello strumento e l'interno della scuola nel quale questa copia si trova.

\newpage
\section{Progettazione concettuale}

\subsection{Lista delle entità}


\textbf{Persona: }modella una generica persona


\begin{itemize}
	\item CodiceFiscale : char(16)
	\item Nome : varchar(20)
	\item Cognome : varchar(20)
	\item NumeroTelefono : numeric(10)
\end{itemize}




\begin{flushleft}
Sono definite le seguenti sottoclassi: (la generalizzazione è parziale, in quanto esiste anche l'entità tecnico ma non compare come sottoclasse)\\
\end{flushleft}


\textbf{Insegnante: }: rappresenta un insegnante della scuola

\begin{description} 
	\item \hspace{2em} \textbf{•} Password: varchar(20)
	\item \hspace{2em} \textbf{•} Conservatorio: varchar(40)
	\item \hspace{2em} \textbf{•} Età: int
\end{description}

\textbf{Allievo: }: rappresenta un allievo della scuola

\begin{description} 
	\item \hspace{2em} \textbf{•} Città: varchar(20)
	\item \hspace{2em} \textbf{•} Età: int
\end{description}

\textbf{Segretario: }: rappresenta un segretario della scuola
\begin{description}
	\item \hspace{2em} \textbf{•} Password: varchar(20)
\end{description}


\begin{flushleft}
\textbf{Interno: }modella un generico ambiente


\begin{itemize}
\item IdLuogo: è un attributo composto da:
	\begin{itemize}
	\item Edificio: char
	\item Piano: smallint
	\item CodAula: char
	\end{itemize}

\end{itemize}

\end{flushleft}



\begin{flushleft}
Sono definite le seguenti sottoclassi: (la generalizzazione è totale)
\end{flushleft}


\textbf{Aula: }: rappresenta un ambiente in cui si svolgono solamente le lezioni \medskip


\textbf{Saletta concerti: }: rappresenta un ambiente in cui si svolgono solamente i saggi

\medskip

\begin{flushleft}
\textbf{Attività: }una generica attività che coinvolge insegnanti e allievi

\begin{itemize}
\item Interno: è un attributo composto da:
	\begin{itemize}
	\item Edificio: char
	\item Piano: smallint
	\item CodAula: char
	\end{itemize}
\item Insegnante: char(16)
\end{itemize}

\end{flushleft}


\begin{flushleft}
Sono definite le seguenti sottoclassi: (la generalizzazione è totale)\\
\end{flushleft}


\textbf{Lezione: }rappresenta un ambiente in cui si svolgono solamente le lezioni %manca identazione
\begin{description}
	\item \hspace{2em} \textbf{•} IdLezione: int 
	\item \hspace{2em} \textbf{•} OraInizio: time
	\item \hspace{2em} \textbf{•} OraFine: time
	\item \hspace{2em} \textbf{•} Allievo: char(16)
	\item \hspace{2em} \textbf{•} SegrPren: char(16)
\end{description}

\medskip


\textbf{Saggio : }rappresenta un ambiente in cui si svolgono solamente i saggi
\begin{description}
	\item \hspace{2em} \textbf{•} DataOraInizio: datetime
\end{description}


\medskip

\begin{flushleft}
\textbf{CopiaStrumento: }modella un esemplare di strumento presente nella scuola
\end{flushleft}

\begin{itemize}
\item IdCopia : char(10)
\item Edificio : char
\item Piano : smallint
\item Aula : char
\end{itemize}


\begin{flushleft}
\textbf{Strumento: }modella una tipologia di strumento presente nella scuola
\end{flushleft}

\begin{itemize}
	\item Nome : varchar(20)
	\item Quantità: int
\end{itemize}


\begin{flushleft}
Sono definite le seguenti sottoclassi: (la generalizzazione è totale)\\
\end{flushleft}



\textbf{Stabili: }rappresenta uno strumento non trasportabile in altri interni \medskip

\textbf{Portatili: }rappresenta uno strumento trasportabile in altri interni della scuola.

\begin{flushleft}
\textbf{Tecnico: }rappresenta un tecnico della scuola
\end{flushleft}

\begin{itemize}
	\item CodiceFiscale : char(16)
	\item NumeroTelefono: numeric(10)
\end{itemize}

\bigskip


\subsection{Lista delle associazioni}\bigskip

\begin{itemize}
\item \textsc {Segretario-Lezione}: \textbf{prenota}
	\begin{itemize}
	\item Un segretario può prenotare nessuna o più lezioni in un giorno, una lezione è prenotata da uno e un solo segretario
	\item Molteplicità 1:n
	\end{itemize}
\end{itemize}

\smallskip

\begin{itemize}
\item \textsc {Allievo-Saggio}: \textbf{partecipa}
	\begin{itemize}
	\item Un allievo può partecipare a nessun o più saggi, al saggio partecipano da uno a n allievi
	\item Molteplicità 1:n
	\end{itemize}
\end{itemize}

\smallskip

\begin{itemize}
\item \textsc {Allievo-Lezione}: \textbf{svolge}
	\begin{itemize}
	\item Un allievo può partecipare a nessuno o più lezioni al giorno, alla lezione partecipa uno e un solo allievo
	\item Molteplicità 1:n
	\end{itemize}
\end{itemize}

\smallskip

\begin{itemize}
\item \textsc {Allievo-Insegnante}: \textbf{istruito}
	\begin{itemize}
	\item Un allievo è istruito da uno o più insegnanti, un insegnante istruisce uno o più allievi
	\item Molteplicità n:n
	\end{itemize}
\end{itemize}

\smallskip

\newpage
\begin{itemize}
\item \textsc {Insegnante-Lezione}: \textbf{tenuta}
	\begin{itemize}
	\item Un insegnante può tenere nessuna o più lezioni al giorno, una lezione è tenuta da uno e un solo insegnante
	\item Molteplicità 1:n
	\end{itemize}
\end{itemize}

\smallskip

\begin{itemize}
\item \textsc {Insegnante-Strumento}: \textbf{insegna}
	\begin{itemize}
	\item Un insegnante insegna uno ed un solo strumento, uno strumento viene insegnato da uno o più insegnanti
	\item Molteplicità 1:n
	\end{itemize}
\end{itemize}

\smallskip

\begin{itemize}
\item \textsc{Insegnante-Saggio}: \textbf{dirige}
	\begin{itemize}
	\item Un insegnante dirige da zero a n saggi, un saggio è diretto da esattamente un insegnante
	\item Molteplicità 1:n
	\end{itemize}
\end{itemize}

\begin{itemize}
\item \textsc {Lezione-Aula}: \textbf{svolta}
	\begin{itemize}
	\item Una lezione è svolta in una ed una sola aula, in un aula si svolgono nessuna o più lezioni
	\item Molteplicità 1:n
	\end{itemize}
\end{itemize}

\smallskip 

\begin{itemize}
\item \textsc {Saggio-Saletta concerti}: \textbf{tenuto}
	\begin{itemize}
	\item Un saggio è tenuto in una ed una sola saletta, in una saletta si possono tenere da zero a più saggi 
	\item Molteplicità 1:n
	\end{itemize}
\end{itemize}

\smallskip

\begin{itemize}
\item \textsc {Interno-Copia stumento}: \textbf{contiene}
	\begin{itemize}
	\item un interno contiene nessuno o più copie strumento, una copia strumento è contenuta in uno e un solo interno
	\item Molteplicità 1:n
	\end{itemize}
\end{itemize}

\smallskip

\begin{itemize}
\item \textsc {Copia strumento-Strumento}: \textbf{copia di}
	\begin{itemize}
	\item una copia strumento si riferisce ad una e una sola copia di strumento, uno strumento ha zero o più copie
	\item Molteplicità 1:n
	\end{itemize}
\end{itemize}

\smallskip

\begin{itemize}
\item \textsc {Tecnico-Strumento}: \textbf{ripara}
	\begin{itemize}
	\item Ad un tecnico è affidata la riparazione di uno ed un solo strumento, uno strumento è riparato da esattamente un tecnico
	\item Molteplicità 1:1
	\end{itemize}
\end{itemize}

\newpage

\thispagestyle{empty}

%per inserire un'immagine
\begin{landscape}
\begin{figure}
\subsection{Schema E-R iniziale del database}
\centering
\includegraphics[angle=0,scale=.40]{projBasi.jpg}
\hspace{1in}
\label{schema}
\caption{schema E-R}
\end{figure}
\end{landscape}

\newpage
\section{Ristrutturazione}\bigskip

\subsection{Analisi delle ridondanze}\medskip

Dall'analisi dello schema concettuale troviamo che:\medskip
\\
\textbf{•} L'attributo \emph{StrumSuonato} dall'entità \textsc{Allievo}
è un attributo ridondante.\\
Esso infatti è deribabile dall' attributo \emph{Strumento} appartenente all'entità \textsc{Insegnante} in quanto si ricorda che       
un insegnante attraverso la relazione INSEGNA ha cardinalità (1,1) verso \textsc{Strumento}. L'attributo \emph{StrumSuonato} viene perciò \textbf{eliminato}.\medskip
\\(Questa scelta è ragionevole in quanto per i requisiti del progetto le query richiedenti gli strumenti suonati dall'allievo non 	sono rilevanti).

\bigskip

\begin{flushleft}
\textbf{•} Gli attributi \emph{Nazione} e \emph{Regione} nell'entità \textsc{Allievo} sono attributi ridondanti in quanto deducibili dall'attributo \emph{Città} presente nella stessa entità.
\\Essi vengono perciò \textbf{eliminati} per il fatto che, oltretutto, non sono necessari ai fini delle query e dei requisiti di questo progetto. 
\end{flushleft}

\bigskip

\begin{flushleft}
\textbf{•} L'attributo \emph{Quantità} nell'entità \textsc{Strumenti} è un attributo ridondante ma, per non dover fare inutili e ripetitive operazioni di conteggio che appensantirebbero solamente il database, esso viene \textbf{mantenuto} per migliorarne l'efficenza.
\end{flushleft}

\bigskip

\begin{flushleft}
\textbf{•} Risultano infine degli ulteriori attributi ridondanti: \emph{Età}, presente nelle entità \textsc{Allievo} ed \textsc{Insegnante} e \emph{NumStrumenti} presente in \textsc{Interni} ma, per i requisiti e per le query eseguite in questo progetto, sono stati \textbf{mantenuti} come attributi separati.
\end{flushleft}

\bigskip

\subsection{Eliminazione delle generalizzazioni}\medskip

\textbf{•} La generalizzazione di \textsc{Persona} si trasforma in tre \textbf{associazioni} uno a uno che legano rispettivamente l'entità genitore \textsc{Persona} con le entità figlie \textsc{Insegnante}, \textsc{Allievo} e \textsc{Segretario}.\\
Non ci sono trasferimenti di attributi o associazioni e le tre entità \textsc{Insegnante}, \textsc{Allievo} e \textsc{Segretario}
sono identificate esternamente dall'attributo \emph{CodiceFiscale} di \textsc{Persona}.\\
Nello schema ottenuto vanno però aggiunti dei vincoli: ogni occorrenza di \textsc{Persona} non può partecipare contemporaneamente
alle relazioni formatesi con le entità figlie.

\bigskip

\begin{flushleft}
\textbf{•} L'entità genitore \textsc{Attività} della generalizzazione è stata \textbf{eliminata} come anche il suo attributo composto \emph{\textbf{Interno}}.
Gli attributi \emph{Piano}, \emph{Edificio} e \emph{Aula} di cui era composto \emph{\textbf{Interno}} sono stati aggiunti alle entità figlie \textsc{Lezione} e \textsc{Saggio}.\smallskip
\\Da segnalare una piccola modifica nell'entità \textsc{Saggio} dove l'attributo \emph{Aula} proveniente dal genitore si chiama ora \emph{Sala}.\smallskip
\\Infine non sono state aggiunte o modificate alcune chiavi primarie e associazioni in quanto \textsc{Attività} non ne possedeva.
\end{flushleft}

\bigskip

\begin{flushleft}
\textbf{•} Dalla generalizzazione di \textsc{Interno} le entità figlie \textsc{Aula} e \textsc{Saletta concerti} vengono \textbf{eliminate} e i loro attributi(in questo caso nessuno) ed associazioni vengono aggiunti all'entità genitore \textsc{Interno}.\\
A tale entità viene aggiunto un ulteriore attributo di nome \emph{Tipologia} che serve a distinguire il "tipo" di occorrenza di un interno, ovvero se si tratta di un'aula o di una saletta concerto.\smallskip
\\In conclusione a seguito di questa fase di ristrutturazione rimarrà solo l'entità \textsc{Interno} con in più l'attributo \emph{Tipologia(Aula,Saletta concerti)}; \textsc{Interno} sarà collegato a \textsc{Lezione} tramite la relazione SVOLTA e a \textsc{Saggio} tramite la relazione TENUTO.
\end{flushleft}

\bigskip

\begin{flushleft}
\textbf{•} Per l'ultima generalizzazione si è proceduto come per \textsc{Interno}.\\
Le entità figlie \textsc{Stabile} e \textsc{Portatile} sono state \textbf{eliminate} e all'entità genitore \textsc{Strumento} è stato unicamente aggiungo un attributo \emph{Tipologia(Stabile,Portatile)} (per distinguere il tipo di strumento) poichè le entità figlie non avevano nè attributi propri nè associazioni con altre entità.
\end{flushleft}

\subsection{Partizionamento/accorpamento di entità e associazioni}\medskip

\textbf{•} Non c'è alcuna operazione di partizionamento.\bigskip
\\
\textbf{•} Abbiamo eseguito invece un accorpamento dell'entità \textsc{Tecnico} all'interno dell'entità \textsc{Strumento} dove quest'ultima avrà quindi ereditato i suoi  attributiText \emph{CodiceFiscale} e \emph{NumeroTelefono} diventati dopo l'accorpamento i nuovi attributi \emph{CodFiscTecnico} e \emph{NumeroTelTecnico}.\\Non vi è alcuna modifica o aggiunta di chiavi primarie.\medskip
\\
Questa ristrutturazione è ritenuta corretta in quanto le operazioni più frequenti sull'entità \textsc{Strumento} richiedono sempre i dati relativi al tecnico che lo ripara e vogliamo quindi risparmiare gli accssi necessari per risalire a tali dati attraverso l'associazione che li lega.\medskip
\\Inoltre non ci sono ovviamente effetti collaterali come creazione di ridondanze, ma neanche una possibile creazione di valori nulli poichè,come già descritto in precendenza nella lista delle associazioni, le cardinalità ci dicono che un tecnico ripara esattamente uno strumento e quello strumento è riparato esattamente da un tecnico quindi ci sono sempre dei valori associati per \emph{CodFiscTecnico} e \emph{NumeroTelTecnico}.

\subsection{Scelta degli identificatori principali}

\underline{Persone}:
\begin{flushleft}
Come identificatore principale è stato scelto \emph{CodiceFiscale} che identifica così in modo univoco
una persona, sia essa segretario, allievo o insegnante.
\end{flushleft}

\begin{flushleft}
Per \underline{Insegnanti}, \underline{Allievi} e \underline{Segretari} la chiava primaria è \emph{CodiceFiscale} che, per ognuna, si riferisce esternamente a \emph{CodiceFiscale} su \underline{Persone}.
\end{flushleft}

\begin{flushleft}
\underline{Interni}:
\end{flushleft}

\begin{flushleft}
Assunto che la scuola sia formata da più edifici, qui la chiave scelta come identificatore è formata da \emph{Edificio}, \emph{Piano}, \emph{Aula} .
\end{flushleft}

\begin{flushleft}
\underline{Lezioni}:
\end{flushleft}

\begin{flushleft}
La chiave primaria è \emph{IdLezione} che identifica univocamente la lezione tramite un codice .
\end{flushleft}

\begin{flushleft}
\underline{Saggi}:
\end{flushleft}

\begin{flushleft}
Qui la chiave primaria che è stata scelta è \emph{DataOraInizio} poichè c'è un solo saggio al giorno quindi questa data sarà differente in ogni tupla.
\end{flushleft}

\begin{flushleft}
\underline{Strumenti}:
\end{flushleft}

\begin{flushleft}
La chiave primaria utilizzata è l'attributo \emph{Nome}.
E' una scelta ragionevole poichè in questa tabella lo strumento è definito solo come concetto e quindi il nome identifica univocamente quello strumento.
\end{flushleft}

\begin{flushleft}
\underline{CopieStrumentiScuola}:
\end{flushleft}

\begin{flushleft}
Qui le copie di uno strumento possono essere molteplici quindi ogni esemplare è giusto che sia identificato tramite un codice univoco. \\L'identicatore principale che è stato scelto è \emph{IdCopia}.
\end{flushleft}

\begin{flushleft}
\underline{Insegna}:
\end{flushleft}

\begin{flushleft}
Insegna è la relazione che definisce ogni tupla tra un insegnante e un allievo e quindi è corretto definire la chiave primaria formata da \emph{Insegnante}, \emph{Allievo}.
\end{flushleft}

\begin{flushleft}
\underline{AllieviAlSaggio}:
\end{flushleft}

\begin{flushleft}
E' la relazione che definisce ogni tupla tra un un allievo e un saggio perciò quindi è corretto definire la chiave primaria formata da \emph{Allievo}, \emph{Saggio}.
\end{flushleft}

\newpage
\thispagestyle{empty}
%per inserire un'immagine
\begin{landscape}
\begin{figure}
\subsection{Schema E-R del database dopo la ristrutturazione}
\centering
\includegraphics[angle=0,scale=.45]{proj2.jpg}
\hspace{1in}
\label{schema}
\caption{schema E-R ristrutturato}
\end{figure}
\end{landscape}

\newpage
\section{Progettazione logica}\bigskip

\subsection{Lista delle tabelle}

\begin{flushleft}
\underline{Persone}: la PK \emph{CodiceFiscale} viene duplicata perchè usata come chiave esterna in altre tabelle.
\end{flushleft}

\begin{itemize}
\item CodiceFiscale: char(16) $\ll$PK$\gg$
\item Nome: varchar(20) not null
\item Cognome: varchar(20) not null
\item NumeroTelefono: numeric(10)
\end{itemize}

\medskip

\begin{flushleft}
\underline{Insegnanti}: viene utilizzato CodiceFiscale come chiave esterna verso \underline{Persone}.
\end{flushleft}

\begin{itemize}
\item CodiceFiscale: char(16) $\ll$PK$\gg$, $\ll$FK(Persone)$\gg$
\item Password: varchar(20)
\item Conservatorio: varchar(40) (nome del conservatorio di provenienza dell'insegnante)
\item Eta: int
\item Strumento: varchar(20) $\ll$FK(Strumenti)$\gg$
\end{itemize}

\medskip

\begin{flushleft}
\underline{Allievi}: viene utilizzato CodiceFiscale come chiave esterna verso \underline{Persone}.
\end{flushleft}

\begin{itemize}
\item CodiceFiscale: char(16) $\ll$PK$\gg$, $\ll$FK(Persone)$\gg$
\item Città: varchar(20)
\item Eta: int
\end{itemize}

\medskip

\begin{flushleft}
\underline{Segretari}: viene utilizzato CodiceFiscale come chiave esterna verso \underline{Persone}.
\end{flushleft}

\begin{itemize}
\item CodiceFiscale: char(16) $\ll$PK$\gg$, $\ll$FK(Persone)$\gg$
\item Password: varchar(20)
\end{itemize}

\medskip

\begin{flushleft}
\underline{Interni}: la PK viene duplicata perchè usata da altre tabelle come chiave esterna.
\end{flushleft}

\begin{itemize}
\item Edificio: char
\item Piano	: smallint
\item CodAula: char
\item Tipologia enum('aula','sala concerti')
\item NumStrumenti: int  (numero di copie presenti nell'interno)
\item MassimoStrumenti: smallint  (campo per memorizzare il massimo numero di strumenti che l'interno può ospitare)
\end{itemize}

$\ll$PK(Edificio,Piano,CodAula)$\gg$

\newpage

\begin{flushleft}
\underline{Lezione}: viene usata la FK(Edificio,Piano,Aula) come chiave esterna verso \underline{Interni}.
\end{flushleft}

\begin{itemize}
\item IdLezione:  int auto\textunderscore increment $\ll$PK$\gg$
\item Insegnante: char(16)
\item Allievo:   char(16)
\item OraInizio:  time     
\item OraFine:    time 	
\item Edificio:   char
\item Piano	:  	 smallint
\item Aula:       char,
\item SegrPren:  char(16) (segretario che ha prenotato la lezione)
\end{itemize}

(Edificio,Piano,Aula) $\ll$FK(Interno)$\gg$

\medskip

\begin{flushleft}
\underline{Saggi}: viene usata la FK(Edificio,Piano,Sala) come chiave esterna verso \underline{Interni}.
\end{flushleft}

\begin{itemize}
\item Insegnante: char(16) $\ll$FK(Insegnanti)$\gg$
\item DataOraInizio: datetime $\ll$PK$\gg$
\item Edificio: char
\item Piano: smallint
\item Sala: char
\end{itemize}

(Edificio,Piano,Sala) $\ll$FK(Interno)$\gg$

\medskip

\begin{flushleft}
\underline{Strumenti}: gli strumenti come concetto, infatti il \emph{Nome} distingue univocamente che strumento è.
\end{flushleft}

\begin{itemize}
\item Nome: varchar(20) $\ll$PK$\gg$
\item Tipologia: enum('portatile','fisso')
\item Quantita: int
\item Tecnico: char(16)
\item NumTelTecnico: numeric(10)	
\end{itemize}

\medskip

\begin{flushleft}
\underline{CopieStrumentiScuola}: uno strumento come esemplare fisico all'interno della scuola.\smallskip \\ \hspace{100pt}  viene usata la FK(Edificio,Piano,Aula) come chiave esterna verso \underline{Interni}.
\end{flushleft}

\begin{itemize}
\item IdCopia: char(10)  $\ll$PK$\gg$
\item Strumento: varchar(20)  $\ll$FK(Strumenti)$\gg$
\item Edificio: char
\item Piano: smallint
\item Aula: char
\end{itemize}

(Edificio,Piano,Aula) $\ll$FK(Interno)$\gg$

\medskip

\begin{flushleft}
\underline{Insegna}:
\end{flushleft}

\begin{itemize}
\item Insegnante char(16) $\ll$FK(Insegnante)$\gg$
\item Allievo char(16) $\ll$FK(Allievi)$\gg$
\end{itemize}

(Insegnante,Allievo) $\ll$PK$\gg$

\medskip

\begin{flushleft}
\underline{AllieviAlSaggio}:
\end{flushleft}

\begin{itemize}
\item Allievo char(16) $\ll$FK(Allievi)$\gg$
\item Saggio datetime $\ll$FK(Saggi)$\gg$
\end{itemize}

 (Allievo,Saggio) $\ll$PK$\gg$



\newpage
\section{Implementazione della base di dati}
\subsection{Creazione tabelle}
\begin{small}
A seguire si allega il codice per la creazione delle tabelle:  \bigskip

%inserimento codice language SQL
\begin{sql}
/*PROBLEMI: NON RIESCO A USARE Edificio, Piano e Aula come chiavi esterne sulla tabella Interni*/

DROP SCHEMA IF EXISTS `gmazzocc-PR`;     /*obbligatorio usare apici inversi a causa del '-'*/
CREATE SCHEMA `gmazzocc-PR`; USE `gmazzocc-PR`;

SET NAMES 'utf8';


/*Contiene i dati di base che hanno Insegnanti, Allievi e Segretari*/
CREATE TABLE Persone (
	CodiceFiscale CHAR(16),
	Nome	      VARCHAR(20) NOT NULL,
	Cognome	      VARCHAR(20) NOT NULL,
	NumeroTelefono NUMERIC(10),
	PRIMARY KEY(CodiceFiscale)
)ENGINE=InnoDB;

/*Si assume che la scuola sia composta da piu' edifici*/
CREATE TABLE Interni(
	Edificio CHAR,
	Piano	 SMALLINT,
	CodAula	 CHAR,
	Tipologia ENUM('aula','sala concerti'), 
	NumStrumenti INT,	/*numero di copie presenti nell'interno*/
	MassimoStrumenti SMALLINT,	/*campo per memorizzare il massimo numero di strumenti che l'Interno puo ospitare*/	
	PRIMARY KEY (Edificio,Piano,CodAula)
)ENGINE=InnoDB;

/*Un tecnico si occupa di un unico tipo di strumenti: e.g. tecnico del pianoforte,tecnico dell'arpa*/
/*DISTINGUERE TRA STRUMENTI COME CONCETTO E 'ESEMPLARE' DI STRUMENTO*/
CREATE TABLE Strumenti(
	Nome 	VARCHAR(20) PRIMARY KEY,
	Tipologia ENUM('portatile','fisso'),
	Quantita INT,          /*e.g. 'quanti pianoforti ci sono nella scuola'*/
	Tecnico CHAR(16),			/*UNIQUE perche' un tecnico non puo' occuparsi di piu' strumenti (era 1:1 nell'ER)*/
	NumTelTecnico NUMERIC(10)		
)ENGINE=InnoDB;

/*Tabella per memorizzare gli esemplari di strumenti presenti nella scuola*/
/*Una copia di uno strumento pu'o trovarsi al piu' in un interno*/
/*In un interno possono trovarsi piu' copie di strumenti*/
CREATE TABLE CopieStrumentiScuola(
	IdCopia   CHAR(10),	/*codice identificativo di uno strumento*/	
	Strumento VARCHAR(20),   
	Edificio  CHAR,       /*in che edificio si trova questa copia*/
	Piano 	  SMALLINT,	/*in che piano si trova questa copia*/
	Aula	  CHAR,		/*in che aula si trova questa copia*/
	PRIMARY KEY (IdCopia),
	FOREIGN KEY (Strumento) REFERENCES Strumenti(Nome) ON DELETE CASCADE,   /*quando viene eliminato uno strumento dalla scuola ==> tutti le copie devono essere 											eliminate*/
	/*Edificio,Piano,Aula sono FOREIGN KEYS su Interni*/
	FOREIGN KEY(Edificio,Piano,Aula) REFERENCES Interni(Edificio,Piano,CodAula) ON DELETE SET NULL
)ENGINE=InnoDB;


/*I segretari possono modificare il database*/
CREATE TABLE Segretari(
	CodiceFiscale CHAR(16),
	Password VARCHAR(20),
	PRIMARY KEY (CodiceFiscale),
	FOREIGN KEY (CodiceFiscale) REFERENCES Persone(CodiceFiscale) ON DELETE CASCADE
)ENGINE=InnoDB;

/*gli insegnanti e gli allievi possono solo visualizzare i dati*/
CREATE TABLE Insegnanti(
	CodiceFiscale CHAR(16),
	Password      VARCHAR(20),
	Conservatorio VARCHAR(40),    /*nome del conservatorio di provenienza dell'insegnante*/
	Eta INT,
	Strumento VARCHAR(20),
	PRIMARY KEY (CodiceFiscale),
	FOREIGN KEY (CodiceFiscale) REFERENCES Persone(CodiceFiscale) ON DELETE CASCADE,
	FOREIGN KEY (Strumento) REFERENCES Strumenti(Nome) ON DELETE SET NULL
)ENGINE=InnoDB;

CREATE TABLE Allievi(
	CodiceFiscale CHAR(16),
	Citta VARCHAR(20),       /*citta' di nascita*/
	Eta INT,
	PRIMARY KEY (CodiceFiscale),
	FOREIGN KEY (CodiceFiscale) REFERENCES Persone(CodiceFiscale) ON DELETE CASCADE
)ENGINE=InnoDB;

CREATE TABLE Insegna(
 	Insegnante CHAR(16),
	Allievo    CHAR(16),
	PRIMARY KEY (Insegnante, Allievo),
	FOREIGN KEY (Insegnante) REFERENCES Insegnanti(CodiceFiscale) ON DELETE CASCADE,
	FOREIGN KEY (Allievo) REFERENCES Allievi(CodiceFiscale) ON DELETE CASCADE
)ENGINE=InnoDB;



/*Un allievo puo' effettuare piu' lezioni nello stesso giorno;
  Un insegnante puo' effettuare piu' lezioni nello stesso giorno;
  (Insegnante DataOraInizio) UNIQUE;
  (Allievi    DataOraInizio) UNIQUE;
*/
CREATE TABLE Lezioni(
	IdLezione  INT AUTO_INCREMENT,
	Insegnante CHAR(16),
	Allievo    CHAR(16),
 	OraInizio  TIME,     
	OraFine    TIME, 	
	Edificio   CHAR,
	Piano	   SMALLINT,
	Aula       CHAR,	
	Segr_Pren  CHAR(16),	 /*segretario che ha prenotato la lezione*/	
	PRIMARY KEY (IdLezione),
	FOREIGN KEY (Insegnante) REFERENCES Insegnanti(CodiceFiscale) ON DELETE CASCADE,
	FOREIGN KEY (Allievo) REFERENCES Allievi(CodiceFiscale) ON DELETE CASCADE,
	FOREIGN KEY (Segr_Pren) REFERENCES Segretari(CodiceFiscale) ON DELETE SET NULL,
	FOREIGN KEY(Edificio,Piano,Aula) REFERENCES Interni(Edificio,Piano,CodAula) ON DELETE CASCADE
)ENGINE=InnoDB;

/*Un insegnante on un allievo possono partecipare a piu' saggi*/
/*Ad un saggio partecipano piu' allievi*/
CREATE TABLE Saggi(
	Insegnante CHAR(16),
	DataOraInizio DATETIME,	/*Chiave perche' un saggio al giorno quindi sempre diversa in ogni tupla*/
	Edificio CHAR,
	Piano	 SMALLINT,
	Sala 	 CHAR,
	PRIMARY KEY (DataOraInizio),
	FOREIGN KEY (Insegnante) REFERENCES Insegnanti(CodiceFiscale) ON DELETE CASCADE,
	FOREIGN KEY(Edificio,Piano,Sala) REFERENCES Interni(Edificio,Piano,CodAula) ON DELETE CASCADE
)ENGINE=InnoDB;



CREATE TABLE AllieviAlSaggio(
	Allievo	CHAR(16),
	Saggio DATETIME,
	PRIMARY KEY(Allievo,Saggio),
	FOREIGN KEY (Allievo) REFERENCES Allievi(CodiceFiscale) ON DELETE CASCADE,
	FOREIGN KEY (Saggio) REFERENCES Saggi(DataOraInizio) ON DELETE CASCADE
)ENGINE=InnoDB;



\end{sql}

\end{small}

\newpage

\section{Query}

\subsection{Query create}

Seguono alcuni esempi di query:\bigskip

\begin{flushleft}
\textbf{1.} Query che estrae gli insegnanti per i quali almeno un allievo ha partecipato ad un saggio tenuto nella sala (A,1,C).
\end{flushleft}

\begin{sql}[float=h]
SELECT P.CodiceFiscale,Nome,Cognome
FROM Persone AS P JOIN Insegnanti AS I ON (P.CodiceFiscale=I.CodiceFiscale)
WHERE EXISTS (SELECT * FROM AllieviAlSaggio AS A_S JOIN Saggi AS S 
		ON (A_S.Saggio=S.DataOraInizio)
		WHERE Insegnante=I.CodiceFiscale AND S.Edificio='A' AND S.Piano=1 AND S.Sala='C');
\end{sql}

\begin{flushleft}
output:
\end{flushleft}

\begin{verbatim}
      +------------------+----------+------------+
      | CodiceFiscale    | Nome     | Cognome    |
      +------------------+----------+------------+
      | mzzgnn94m08a703j | Giovanni | Mazzocchin |
      +------------------+----------+------------+
\end{verbatim}

\bigskip

\begin{flushleft}
\textbf{2.} Query che estrae i Codice fiscale e i Nome degli allievi che hanno tutti gli insegnanti di età superiore ai 50 anni.
\end{flushleft}

\begin{sql}[float=h]
SELECT P.CodiceFiscale, Nome 
FROM Persone AS P JOIN Allievi AS A ON (P.CodiceFiscale=A.CodiceFiscale) 
WHERE NOT EXISTS (SELECT * FROM Insegna AS I1 JOIN Insegnanti AS I2 ON (I1.Insegnante=I2.CodiceFiscale)
		  WHERE (I1.Allievo = P.CodiceFiscale AND I2.Eta <= 50)				
		 );
\end{sql}

\begin{flushleft}
output:
\end{flushleft}

\begin{verbatim}
      +------------------+---------+
      | CodiceFiscale    | Nome    |
      +------------------+---------+
      | zccgcm93d04g693a | Giacomo |
      | zccgcq93d04g691a | Giacomo |
      | zkcgcm93d04g692a | Paolo   |
      +------------------+---------+
\end{verbatim}

\newpage

\begin{flushleft}
\textbf{3.} Query che estrae l'allievo che ha partecipato a più saggi (per farlo si crea una view di appoggio).
\end{flushleft}

\begin{sql}[float=h]
DROP VIEW IF EXISTS NumeroSaggiPerAllievo;
CREATE VIEW NumeroSaggiPerAllievo AS (SELECT A1.CodiceFiscale,COUNT(*) AS NumeroSaggi
				      FROM Allievi AS A1,AllieviAlSaggio AS A2
				      WHERE (A1.CodiceFiscale=A2.Allievo)
				      GROUP BY A1.CodiceFiscale);
SELECT N.CodiceFiscale,Nome,Cognome 
FROM NumeroSaggiPerAllievo AS N JOIN Persone AS P ON (N.CodiceFiscale=P.CodiceFiscale)
WHERE NumeroSaggi >= ALL(SELECT NumeroSaggi FROM NumeroSaggiPerAllievo); 
\end{sql}

\begin{flushleft}
output:
\end{flushleft}

\begin{verbatim}
      +------------------+-------+---------+
      | CodiceFiscale    | Nome  | Cognome |
      +------------------+-------+---------+
      | RSSMRA92A01H501Y | Mario | Rossi   |
      +------------------+-------+---------+
\end{verbatim}

\bigskip

\begin{flushleft}
\textbf{4.} Query che estrae DataOraInizio e Edificio,Piano e Sala del saggio a cui hanno partecipato
più allievi (anche qui si crea una view di appoggio).
\end{flushleft}

\begin{sql}[float=h]
DROP VIEW IF EXISTS NumeroAllieviPerSaggio;
CREATE VIEW NumeroAllieviPerSaggio AS (SELECT S.DataOraInizio,S.Edificio,S.Piano,S.Sala,COUNT(*) AS NumeroAllievi
						FROM Saggi AS S,AllieviAlSaggio AS S1
						WHERE (S.DataOraInizio=S1.Saggio)
						GROUP BY DataOraInizio
				      );
SELECT DataOraInizio,Edificio,Piano,Sala,NumeroAllievi
FROM NumeroAllieviPerSaggio
WHERE NumeroAllievi >= ALL (SELECT NumeroAllievi FROM NumeroAllieviPerSaggio);
\end{sql}

\begin{flushleft}
output:
\end{flushleft}

\begin{verbatim}
      +---------------------+----------+-------+------+---------------+
      | DataOraInizio       | Edificio | Piano | Sala | NumeroAllievi |
      +---------------------+----------+-------+------+---------------+
      | 2015-01-01 21:00:01 | A        |     1 | C    |             4 |
      +---------------------+----------+-------+------+---------------+
\end{verbatim}

\newpage

\begin{flushleft}
\textbf{5.} Query che estrae gli Nome e NumeroCopie degli strumenti dei quali sono presenti almeno 5 copie (view di appoggio).
\end{flushleft}

\begin{sql}[float=h]
DROP VIEW IF EXISTS NumeroCopiePerStrumento;
CREATE VIEW NumeroCopiePerStrumento AS
  (SELECT Nome,COUNT(*) AS NumeroCopie
   FROM Strumenti AS S,CopieStrumentiScuola AS C
   WHERE (S.Nome = C.Strumento)
   GROUP BY Nome
);

SELECT Nome,NumeroCopie 
FROM  NumeroCopiePerStrumento
WHERE (NumeroCopie >= 5);
\end{sql}

\begin{flushleft}
output:
\end{flushleft}

\begin{verbatim}
      +------------+-------------+
      | Nome       | NumeroCopie |
      +------------+-------------+
      | pianoforte |           5 |
      +------------+-------------+
\end{verbatim}

\bigskip

\begin{flushleft}
\textbf{6.} Query che estrae l'aula che ha più pianoforti (view di appoggio).

\end{flushleft}

\begin{sql}[float=h]
DROP VIEW IF EXISTS NumPianofortiPerInterno;
CREATE VIEW NumPianofortiPerInterno AS	
SELECT I.Edificio,I.Piano,I.CodAula,COUNT(*) AS NumeroPianoforti
FROM CopieStrumentiScuola AS C JOIN Interni AS I ON (C.Edificio=I.Edificio AND C.Piano=I.Piano AND C.Aula=I.CodAula)
WHERE C.Strumento='pianoforte'
GROUP BY I.Edificio,I.Piano,I.CodAula;

SELECT * FROM NumPianofortiPerInterno AS N
WHERE NumeroPianoforti >= ALL (SELECT NumeroPianoforti FROM NumPianofortiPerInterno);
\end{sql}

\begin{flushleft}
output:
\end{flushleft}

\begin{verbatim}
      +----------+-------+---------+------------------+
      | Edificio | Piano | CodAula | NumeroPianoforti |
      +----------+-------+---------+------------------+
      | A        |     2 | B       |                3 |
      +----------+-------+---------+------------------+
\end{verbatim}

\newpage

\section{Funzioni e Procedure}

\subsection{Funzioni e procedure utilizzate}

Seguono alcuni esempi :\bigskip

\begin{flushleft}
\textbf{1.} \textbf{Funzione} che, passandole il CodiceFiscale di un insegnante, ritorna l'età media dei suoi allievi.
\end{flushleft}

\begin{sql}[float=h]
DROP FUNCTION IF EXISTS EtaMediaClasse;

DELIMITER //

CREATE FUNCTION EtaMediaClasse (p_CodFisc CHAR(16)) RETURNS INT
BEGIN
	DECLARE etaMedia INT;
	SELECT AVG(Eta) 
	FROM Insegna JOIN Allievi ON (Allievo = CodiceFiscale) 
	WHERE (Insegnante = p_CodFisc)
	GROUP BY Insegnante
	INTO etaMedia;

 	RETURN etaMedia;
END //;
\end{sql}

\bigskip

\begin{flushleft}
\textbf{2.} \textbf{Procedure} per l'inserimento di un allievo.\\
Essa riceve i dati di un allievo (propri della tabella \textsc{Allievi} e i suoi insegnanti (al massimo 3) e produce l'aggiunta di un'ennupla alla tabella \textsc{Allievi}, un'ennupla alla tabella \textsc{Persone} e  delle ennuple (massimo 3) alla tabella \textsc{Insegna}.
\end{flushleft}

\begin{sql}
DROP PROCEDURE IF EXISTS InserisciAllievo;

SET NAMES 'utf8';

DELIMITER //

CREATE PROCEDURE InserisciAllievo 
	(
		
		IN p_CodiceFiscale CHAR(16),		
		IN p_Nome VARCHAR(20),
		IN p_Cognome VARCHAR(20),
		IN p_NumeroTelefono NUMERIC(10),
		IN p_Citta   VARCHAR(20),
		IN p_Eta	INT,
		IN p_Insegnante1 CHAR(16),  
		IN p_Insegnante2 CHAR(16),	/*se l'insegnante non esiste ==> trigger che impedisce l'inserimento della tupla*/
		IN p_Insegnante3 CHAR(16)
	)
BEGIN
		INSERT INTO Persone(
			CodiceFiscale,
			Nome,
			Cognome,
			NumeroTelefono
		)
		VALUES (
			p_CodiceFiscale,
			p_Nome,
			p_Cognome,
			p_NumeroTelefono
		);
	        INSERT INTO Allievi
		   (
			CodiceFiscale,
			Citta,
			Eta
		   ) 
		   VALUES 
		   (
			p_CodiceFiscale,
			p_Citta,
			p_Eta
	           );
/*aggiunge alla tabella Insegna solo se l'insegnante passato esiste*/
	       IF p_Insegnante1 <> 'NULL' && p_Insegnante1 <> '' THEN
			INSERT INTO Insegna(Insegnante,Allievo)
                                VALUES(p_Insegnante1,p_CodiceFiscale);			
	       END IF;
	       IF p_Insegnante2 <> 'NULL' && p_Insegnante2 <> ''  THEN
			INSERT INTO Insegna(Insegnante,Allievo)
                                VALUES(p_Insegnante2,p_CodiceFiscale);			
	       END IF;       
	       IF p_Insegnante3 <> 'NULL' && p_Insegnante3 <> '' THEN
			INSERT INTO Insegna(Insegnante,Allievo)
                                VALUES(p_Insegnante3,p_CodiceFiscale);			
	       END IF;
END  //

DELIMITER ;
\end{sql}

\bigskip

\begin{flushleft}
\textbf{3.} \textbf{Procedure} per inserire un nuovo insegnante.
\end{flushleft}

\begin{sql}
DROP PROCEDURE IF EXISTS InserisciInsegnante;

SET NAMES 'utf8';

DELIMITER //

CREATE PROCEDURE InserisciInsegnante(
			IN p_CodiceFiscale CHAR(16),
			IN p_Password	   VARCHAR(20),
			IN p_Nome	   VARCHAR(20),
			IN p_Cognome	   VARCHAR(20),
			IN p_NumeroTelefono NUMERIC(10),
			IN p_Conservatorio VARCHAR(40),
			IN p_Eta	   INT,
			IN p_Strumento	   VARCHAR(20)	
		)
BEGIN
	DECLARE ExistsInstr VARCHAR(20);
	SELECT Nome FROM Strumenti WHERE (Nome=p_Strumento) INTO ExistsInstr;
	IF ExistsInstr IS NOT NULL THEN
		INSERT INTO Persone(
			CodiceFiscale,
			Nome,
			Cognome,
			NumeroTelefono
		)
		VALUES(
			p_CodiceFiscale,
			p_Nome,
			p_Cognome,
			p_NumeroTelefono
		);
                INSERT INTO Insegnanti
                   (
                        CodiceFiscale,
                        Password,
                        Conservatorio,
			Eta,
			Strumento
                   )
                   VALUES
                   (
                        p_CodiceFiscale,
			p_Password,
                        p_Conservatorio,
                        p_Eta,
			p_Strumento
                   );
	END IF;
END //
\end{sql}

\bigskip

\begin{flushleft}
\textbf{4.} Questa \textbf{procedure} riceve come parametri il CodiceFiscale dell'insegnante che sta 
cancellando un proprio allievo (che ha fatto il login nell'area privata) e il codice fiscale dell'allievo che si vuole eliminare.\\
Dopo il DELETE sulla tabella \textsc{Insegna}, bisogna eliminare tutte le lezioni e i saggi in cui
sono coinvolti sia l'insegnante sia l'allievo.
\end{flushleft}

\begin{sql}
DROP PROCEDURE IF EXISTS EliminaAllievo;

SET NAMES 'utf8';

DELIMITER //

CREATE PROCEDURE EliminaAllievo(
	           p_CodiceInsegnante CHAR(16),
		   p_CodiceAllievo    CHAR(16)
		)
BEGIN	
	    DELETE FROM Insegna WHERE (Insegnante=p_CodiceInsegnante AND Allievo=p_CodiceAllievo);
	    DELETE FROM Lezioni WHERE (Insegnante=p_CodiceInsegnante AND Allievo=p_CodiceAllievo);
/*selezionare i saggi dell insegnante con p_CodiceInsegnante*/

END //
DELIMITER ;
\end{sql}

\bigskip

\begin{flushleft}
\textbf{5.} \textbf{Procedure} per inserire un saggio.
\end{flushleft}

\begin{sql}
DROP PROCEDURE IF EXISTS InserisciSaggio;

SET NAMES 'utf8';

DELIMITER //

CREATE PROCEDURE InserisciSaggio(
				p_Insegnante CHAR(16),
				p_DataOra DATETIME,
				p_Edificio CHAR,
				p_Piano    SMALLINT,
				p_Sala	   CHAR,
				p_Allievo1 CHAR(16),
				p_Allievo2 CHAR(16),
				p_Allievo3 CHAR(16),
				p_Allievo4 CHAR(16)
			)
BEGIN
		INSERT INTO Saggi(Insegnante,DataOraInizio,Edificio,Piano,Sala)
			VALUES (p_Insegnante,p_DataOra,p_Edificio,p_Piano,p_Sala);
 
		/*insert on AllieviAlSaggio only if Allievo is not null*/	
		IF p_Allievo1 <> 'NULL' && p_Allievo1 <> '' THEN 
			INSERT INTO AllieviAlSaggio(Allievo,Saggio)
                                VALUES(p_Allievo1,p_DataOra);			
		END IF;
		IF p_Allievo2 <> 'NULL' && p_Allievo2 <> '' THEN
                        INSERT INTO AllieviAlSaggio(Allievo,Saggio)
                                VALUES(p_Allievo2,p_DataOra);
                END IF;
		IF p_Allievo3 <> 'NULL' && p_Allievo3 <> '' THEN
                        INSERT INTO AllieviAlSaggio(Allievo,Saggio)
                                VALUES(p_Allievo3,p_DataOra);
                END IF;
		IF p_Allievo4 <> 'NULL' && p_Allievo4 <> '' THEN
                        INSERT INTO AllieviAlSaggio(Allievo,Saggio)
                                VALUES(p_Allievo4,p_DataOra);
                END IF;      
END //
DELIMITER ;

\end{sql}

\bigskip

\begin{flushleft}
\textbf{6.} \textbf{Procedure} per prenotare una nuova lezione.
\end{flushleft}

\begin{sql}
DROP PROCEDURE IF EXISTS PrenotaLezione;

SET NAMES 'utf8'

DELIMITER //

CREATE PROCEDURE PrenotaLezione(
			p_Insegnante 	CHAR(16),
			p_Allievo  	CHAR(16), 
			p_OraInizio 	TIME,
			p_OraFine   	TIME,
			p_Edificio  	CHAR,
			p_Piano		SMALLINT,
			p_Aula		CHAR,
			p_Segr_Pren	CHAR(16)
		)
BEGIN
		INSERT INTO 
		    Lezioni(IdLezione,Insegnante,Allievo,OraInizio,OraFine,Edificio,Piano,
				Aula,Segr_Pren)
		VALUES (NULL,p_Insegnante,p_Allievo,p_OraInizio,                 /*null perche'' AUTO_INCREMENT*/
				p_OraFine,p_Edificio,p_Piano,p_Aula,p_Segr_Pren);
END //
DELIMITER ;
\end{sql}




\newpage
\section{Triggers}

\subsection{Triggers implementati}

Seguono alcuni esempi di trigger realizzati:\bigskip

\begin{flushleft}
\textbf{1.} \textbf{Trigger} che elimina un saggio quando il numero di allievi che a esso partecipano diventa 0.
\end{flushleft}

\begin{sql}
DROP TRIGGER IF EXISTS AnnullaSaggio;

DELIMITER //
CREATE TRIGGER AnnullaSaggio
AFTER DELETE ON AllieviAlSaggio
FOR EACH ROW
BEGIN
       	DECLARE zeroPupilConcertDate DATETIME;       /*variabile che memorizza il codice del saggio con zero allievi*/
	SELECT DataOraInizio INTO zeroPupilConcertDate
        FROM NumeroAllieviPerSaggio 
	WHERE NumeroAllievi = 0;

 	DELETE FROM Saggi 
	WHERE DataOraInizio = zeroPupilConcertDate;	
END //
DELIMITER ;

\end{sql}

\bigskip

\begin{flushleft}
\textbf{2.} Quando viene aggiunta una copia di uno strumento questo \textbf{trigger}:\\
a) Incrementa di 1 il numero di strumento presenti nell' aula in cui è stato inserita la nuova copia.\\
b) Incrementa di 1 la quantità di quello strumento (nella tabella Strumenti).
\end{flushleft}

\begin{sql}
DROP TRIGGER IF EXISTS UpdateNumInstr;

DELIMITER //
CREATE TRIGGER UpdateNumInstr
AFTER INSERT ON CopieStrumentiScuola
FOR EACH ROW
BEGIN
	/*controllare che il numero di strumenti nell Interno non sia gia' massimo*/
	DECLARE numStr_beforeInsert INT;
	DECLARE max_strumenti INT;
	SELECT NumStrumenti FROM Interni 
	WHERE (Edificio = NEW.Edificio AND Piano=NEW.Piano AND CodAula=NEW.Aula) INTO numStr_beforeInsert;
	/*aggiorna il numero di strumenti solo se numStr_beforeInsert < MassimoStrumenti*/
	SELECT MassimoStrumenti FROM Interni
        WHERE (Edificio = NEW.Edificio AND Piano=NEW.Piano AND CodAula=NEW.Aula) INTO max_Strumenti;
	
	IF numStr_beforeInsert < max_Strumenti THEN
		UPDATE Interni
		SET NumStrumenti = NumStrumenti + 1
		WHERE (Edificio = NEW.Edificio AND Piano=NEW.Piano AND CodAula=NEW.Aula);
	END IF;	
	/*aggiorna il numero di strumenti di quel tipo*/
	UPDATE Strumenti
	SET Quantita = Quantita + 1
	WHERE (Nome = NEW.Strumento);
END //
DELIMITER ;
\end{sql}

\bigskip

\begin{flushleft}
\textbf{3.} \textbf{Trigger} che controlla che un insegnante non superi mai le 4 lezioni al giorno. Se ciò accade, generare un error
\end{flushleft}

\begin{sql}
DROP TRIGGER IF EXISTS MaxLessonsForTeacher;

DELIMITER //
CREATE TRIGGER MaxLessonsForTeacher
BEFORE INSERT ON Lezioni
FOR EACH ROW
BEGIN
	DECLARE numLessons INT;
	SELECT COUNT(*)
	FROM Lezioni
	WHERE Insegnante=NEW.Insegnante
	INTO numLessons;
	IF numLessons > 3  THEN
	/*RAISE AN ERROR*/	
		UPDATE Lezioni SET attr = 'error';
	END IF; 
END //
DELIMITER ;
\end{sql}

\bigskip

\section{Interfaccia Web}

Di seguito alcuni snippet più interessanti relativi alla parte web:

\subsection{index.html}

È la home del sito della scuola; contiene le prime e fondamentali informazioni che appaiono immediatamente davanti a chi vi accede.

\bigskip

\begin{html}
<!DOCTYPE html PUBLIC "-//W3C//DTD XHTML 1.0 Strict//EN" "http://www.w3.org/TR/xhtml1/DTD/xhtml1-strict.dtd">
<html xmlns="http://www.w3.org/1999/xhtml">
	<head>
		<meta http-equiv="Content-Type" content="text/html; charset=utf-8" />
		<meta name="author" content="Giovanni Mazzocchin,Giacomo Zecchin"></meta>
		<meta name="description" content="Homepage sito Scuola di musica privata di Padova"/>
		<meta name="keywords" content="scuola di musica,strumenti musicali"/>
		<title>Scuola di musica</title>
		<link rel="stylesheet" type="text/css" href="css/defaultStyle.css" />
	</head>
	<body>
		<div id="header">
			<h1><span>Scuola di Musica</span></h1>
			</div>
			<div id="path">
				<p>Ti trovi in: <span xml:lang="en" id="breadcrumb">Home</span></p>
			</div>

		<div id="menu">
			<ul>
				<li id="currentLink"><span xml:lang="en">Home</span></li>
				<li><a href="instruments.php">Cosa insegniamo</a></li>     <!--links to public pages-->
				<li><a href="aule.php">Aule</a></li>     <!--links to public pages-->
				<li><a href="lezioni.html">Lezioni odierne</a></li>
				<li><a href="saggi.php">Saggi</span></a></li>
				<li><a href="contatti.html">Contatti</a></li>
				<li><a href="signin.html">Area privata 1</a></li>
				<li><a href="signin_1.html">Area privata 2</a></li>
			</ul>
		</div>
		<div id="content">
			<h1>La scuola</h1>
			<p> Benvenuti nella nostra scuola!
			</p>
			<img src="images/violino.jpg" alt="Il violino e' uno degli strumenti insegnati"/>
		</div>
		<div id="footer">
			</span>
			<span id="w3c">
				<a href="http://validator.w3.org/check?uri=referer"><img src="http://www.w3.org/Icons/valid-xhtml10" alt="Valid XHTML 1.0 Strict"/></a>
				<a href="http://jigsaw.w3.org/css-validator/check/referer"><img src="http://jigsaw.w3.org/css-validator/images/vcss" alt="CSS Valido"/></a>
    			<a href="http://www.w3.org/WAI/WCAG2AAA-Conformance" title="Explanation of WCAG 2.0 Level Triple-A Conformance"><img src="http://www.w3.org/WAI/wcag2AAA" alt="Level Triple-A conformance"/></a>
			</span>
		</div>
	</body>
</html>
\end{html}

\bigskip


\subsection{add\textunderscore lesson.php}

È lo snippet php che serve ad aggiungere una lezione e contiene tutti i campi ad essa necessari: allievi ed insegnanti coinvoltici, segretario che l'ha prenotata, piano, edificio, aula dell'interno nel quale si svolge e gli orari di inizio e fine. L'id della lezione è di tipo auto\textunderscore increment quindi automaticamente sarà sempre diverso per ogni lezione aggiunta.

\bigskip

\begin{php}
<?php
session_start();
if(!isset($_SESSION['codiceFiscale'])) {
    header("Location: index.html");
}
?>

<!DOCTYPE html PUBLIC "-//W3C//DTD XHTML 1.0 Strict//EN" "http://www.w3.org/TR/xhtml1/DTD/xhtml1-strict.dtd">
<html xmlns="http://www.w3.org/1999/xhtml" xml:lang="it" lang="it">
        <head>
                <meta http-equiv="Content-Type" content="text/html; charset=utf-8"></meta>
                <meta name="author" content="Giovanni Mazzocchin,Giacomo Zecchin"></meta>
                <meta name="description" content="Form di login"></meta>
                <meta name="keywords" content="scuola di musica, strumenti"></meta>
                <title>Scuola di musica - Area privata</title>
                <link rel="stylesheet" type="text/css" href="css/style.css" media="screen"></link>
                <link rel="stylesheet" type="text/css" href="css/small.css" media="handheld, screen and (max-width:720px)"></link>
		<script type="text/javascript" src="js/scripts.js"></script>  
      </head>
        <body>
                <p><a tabindex="1" href="#content" class="navHelp">Vai al contenuto</a></p>
                <div id="header">
                        <h1>Scuola di musica</h1>
                </div>
                <div id="path">
                        <p>Ti trovi in: <span xml:lang="en" id="clb">Area privata</span></p>
                </div>

                 <div id="menu">
                        <h1><span>Menu area riservata</span></h1>
                        <ul>
                                <li><a tabindex="2" href="privateHomeSegr.php">Home riservata-Segretari</a></li>
                                <li><a tabindex="4" href="add_pupil.php">Inserire nuovo allievo</a></li>
                                <li><a tabindex="4" href="add_teacher.php">Inserire nuovo insegnante</a></li>
                                <li id="CurrentLink">Prenotare una lezione</a></li>
                <li><a tabindex="5" href="logout.php">Logout</a></li>
                        </ul>
                </div>

                <div id="content">
                        <form name="AddLesson" id="addLessonForm" method="POST" action="check_addlesson.php">
                                <h1><span lang="en">Inserisci lezione</span></h1>
                                <fieldset>
                                        <label for="insegnante">Codice fiscale insegnante</label>
                                          <input type="text" name="insegnante" id="insegnante" tabindex="6"></input>
					<label for="allievo">Codice fiscale allievo</label>
                                          <input type="text" name="allievo" id="allievo" tabindex="7"></input>
                                        <label for="oraInizio">Orario di inizio (hh:mm)</label>
					  <input type="text" name="oraInizio" id="oraInizio" tabindex="8"></input>
					<label for="oraFine">Ora di fine (hh:mm)</label>
                                          <input type="text" name="oraFine" id="oraFine" tabindex="9"></input>
					<label for="edificio">Edificio</label>
					  <input type="text" name="edificio" id="edificio" tabindex="10"></input>
					<label for="piano">Piano</label>
					  <input type="text" name="piano" id="piano" tabindex="11"></input>
					<label for="aula">Aula</label>
					  <input type="text" name="aula" id="aula" tabindex="12"></input>
					<input type="submit" name="submit" id="submit" value="Invia" tabindex="13"></input>
                                </fieldset>
                        </form>
                </div>
                <p><a tabindex="14" href="#menu" class="navHelp">Torna al menu</a></p>
                <div id="footer">
                        <span id="credits">Developed by JazzFunTeam</span>
                        <span id="w3c">
                                <a class="popup" href="http://validator.w3.org/check?uri=referer"><img src="http://www.w3.org/Icons/valid-xhtml10" alt="Valid XHTML 1.0 Strict" /></a>

                                <a class="popup" href="http://jigsaw.w3.org/css-validator/check/referer"><img src="http://jigsaw.w3.org/css-validator/images/vcss" alt="CSS Valido!"/></a>

                        <a class="popup" href="http://www.w3.org/WAI/WCAG2AAA-Conformance" title="Explanation of WCAG 2.0 Level Triple-A Conformance"><img src="http://www.w3.org/WAI/wcag2AAA" alt="Level Triple-A conformance"/></a>
                        </span>
                </div>
        </body>
</html>


\end{php}

\bigskip

\newpage
\subsection{check\textunderscore addlesson.php}

Si è stabilito di implementare il controllo sui conflitti nello scheduling delle lezioni tramite l'applicazione web.\\
Questo form php fa proprio quello che è stato appena descritto.

\bigskip

\begin{php}
<?php
session_start();
include("config.php");

/*1)riceve i dati da 'add_lesson.html'*/
/*2)si connette al database*/
/*3)chiama una procedure SQL per l'inserimento*/

connect_to_DB('gmazzocc','8UBuFcp1');
mysql_query('SET foreign_key_checks = 0');

$insegnante=filter_var($_POST['insegnante'],FILTER_SANITIZE_STRING);
$allievo=filter_var($_POST['allievo'],FILTER_SANITIZE_STRING);
$oraInizio=filter_var($_POST['oraInizio'],FILTER_SANITIZE_STRING);
$oraFine=filter_var($_POST['oraFine'],FILTER_SANITIZE_STRING);
$edificio=filter_var($_POST['edificio'],FILTER_SANITIZE_STRING);
$piano=filter_var($_POST['piano'],FILTER_SANITIZE_STRING);
$aula=filter_var($_POST['aula'],FILTER_SANITIZE_STRING);
//il Codice fiscale del segretario che ha prenotato la lezione
//viene dedotto dal login

//BISOGNA OTTENERE IL CODICE FISCALE DEL SEGRETARIO LOGGATO
$logged_segr =  $_SESSION['codiceFiscale'];

$insegnante = removeSpaces($insegnante);
$allievo = removeSpaces($allievo);
$oraInizio = removeSpaces($oraInizio);
$oraFine = removeSpaces($oraFine);
$edificio = removeSpaces($edificio);
$piano = removeSpaces($piano);
$aula = removeSpaces($aula);

//NB: NESSUN CAMPO PUO' ESSERE VUOTO
if (empty($insegnante) or empty($allievo) or empty($oraInizio) or empty($oraFine)
	or empty($edificio) or empty($piano) or empty($aula)){
	echo "Nessun campo puo' essere vuoto" . "\n";
   	header('Refresh: 3; url= add_lesson.php');
}
/*controlla che insegnante e allievo siano codici di 16 caratteri*/
else if (strlen($insegnante)!=16 or strlen($allievo)!=16){
	echo "Inserire codici fiscali validi" . "\n";
	header('Refresh: 3; url= add_lesson.php');
}

else if (strlen($edificio)>1 or strlen($aula)>1){
	echo "Inserire un interno valido" . "\n";
	header('Refresh: 3; url= add_lesson.php');
}

else if (isHour($oraInizio)==0 or isHour($oraFine)==0){
	echo "Inserire orari validi" . "\n";
	header('Refresh: 3; url= add_lesson.php');
}

else if ($oraFine <= $oraInizio){
	echo "L'orario di fine deve essere successivo a quello di inizio" . "\n";
	header('Refresh: 3; url= add_lesson.php');
}


/*CONTROLLO DEI CONFLITTI:
  per i dati della lezione inseriti nel form, controllare conflitti su
	1)Aula
	2)Insegnante
	3)Allievo
 Quindi aumentare un contatore ogni volta che non c'e' un conflitto ==>
 Alla fine, se il contatore == 3 ==> inserisce lezione nel db
  1) Query su tabella Lezioni con Edificio,Piano,Aula = EdificioInserito,PianoInserito,AulaInserita
	e (OraInizioInserita >= OraInizio AND OraInizioInserita <= OraFine)
	   OR (OraFineInserita >= OraInizio AND OraFineInserita <= OraFine);
     Se la query precedente ha restituisce 1 riga ==> ERRORE('Aula occupata')
     Altrimenti, passare alla query successiva e aumentare il contatore
  2) Query su tabella Lezioni con Insegnante = InsegnanteInserito
	e (OraInizioInserita >= OraInizio AND OraInizioInserita <= OraFine)
           OR (OraFineInserita >= OraInizio AND OraFineInserita <= OraFine)
     Se la query precedente restituisce 1 riga ==> ERRORE('Insegnante impegnato')
     Altrimenti, passare alla query successiva
  3) Query su tabella Lezioni con Allievo = AllievoInserito
	e (OraInizioInserita >= OraInizio AND OraInizioInserita <= OraFine)
           OR (OraFineInserita >= OraInizio AND OraFineInserita <= OraFine)
     Se la query precedente restituisce 1 riga ==> ERRORE('Insegnante impegnato')  
     Altrimenti, aumentare il contatore e inserire la lezione nel db
    
*/
 
/*
AND (('$oraInizio'>=OraInizio AND '$oraInizio'<=OraFine)
                        OR ('$oraFine'>=OraInizio AND '$oraFine'<=OraFine)))
*/
else{
	/*NB: a)controllare che l'insegnante inserito e l'allievo inserito siano nella relazione 'Insegna'	
	      b)controllare che l'edificio,il piano e l'aula inseriti rappresentino effettivamente un'aula*/
	$checkRel_teach_pupil = mysql_query("SELECT * FROM Insegna WHERE Insegnante='$insegnante' AND Allievo='$allievo'");
	if(!$checkRel_teach_pupil){
		die(mysql_error());
	}	
	$check_is_aula = mysql_query("SELECT * FROM Interni WHERE Edificio='$edificio' AND Piano='$piano' AND CodAula='$aula'
					AND Tipologia='aula'");
	if(!$check_is_aula){
		die(mysql_error());
	}
	
	if (mysql_num_rows($checkRel_teach_pupil)==1 and mysql_num_rows($check_is_aula)==1){
		$res1 = mysql_query("SELECT * FROM Lezioni WHERE ((Edificio='$edificio' AND Piano='$piano'
                	AND Aula='$aula') AND (('$oraInizio'>=OraInizio AND '$oraInizio'<=OraFine)
                	        OR ('$oraFine'>=OraInizio AND '$oraFine'<=OraFine))) ");
		if (mysql_num_rows($res1) >= 1){
			echo "Impossibile inserire la lezione,aula occupata" . "\n";
			header('Refresh: 3; url= add_lesson.php');
		}

		else{
			$query2 =  "SELECT * FROM Lezioni WHERE (Insegnante='$insegnante'
		       		AND (('$oraInizio'>=OraInizio AND '$oraInizio'<=OraFine)
					   OR ('$oraFine'>=OraInizio AND '$oraFine'<=OraFine)))";
			$res2 = mysql_query($query2);
			if (mysql_num_rows($res2) >= 1){
				 echo "Impossibile inserire la lezione,insegnante impegnato" . "\n";
			 	 header('Refresh: 3; url= add_lesson.php');
			}
			else{
				$query3 =  "SELECT * FROM Lezioni WHERE (Allievo='$allievo'
				  	AND (('$oraInizio'>=OraInizio AND '$oraInizio'<=OraFine)
                        	                OR ('$oraFine'>=OraInizio AND '$oraFine'<=OraFine)))";
				$res3 = mysql_query($query3);
				if (mysql_num_rows($res3) >= 1){
					echo "Impossibile inserire la lezione,allievo impegnato" . "\n";
			                header('Refresh: 3; url= add_lesson.php');
				}
				else{
				//se sono qui significa che non c'e' alcun conflitto		
					$query = mysql_query("CALL PrenotaLezione('$insegnante','$allievo','$oraInizio',
					'$oraFine','$edificio','$piano','$aula','$logged_segr')");      
					if ($query == FALSE){
						die(mysql_error());
					}
				}
			}
		}
	}
	else if (mysql_num_rows($checkRel_teach_pupil)==0){
		echo "L'allievo inserito non appartiene alla classe dell'insegnante inserito" . "\n";
		header('Refresh: 3; url= add_lesson.php');
	}
	else if (mysql_num_rows($check_is_aula)==0){
		echo "L'aula inserita non esiste oppure non e' adibita per le lezioni" . "\n";
		header('Refresh: 3; url= add_lesson.php');
	}
}
?>	

\end{php}

\bigskip

\newpage
\section{Utilizzo del progetto}

\subsection{Accesso alla parte privata}

Come specificato dalla descrizione dei requisiti del progetto, solamente gli \textbf{insegnanti} ed i \textbf{segretari} possiedono una propria \textbf{area personale}, dove posso eseguire le operazioni per le quali sono autorizzati.

\begin{flushleft}
Di seguito ne elenchiamo, per comodità d'utilizzo, \underline{codici fiscali} e \underline{password}:
\end{flushleft}

\bigskip

\begin{verbatim}
Insegnanti:

+------------------+----------+   
| CodiceFiscale    | Password |
+------------------+----------+
| FSLFNC94S07G693X | pass     |
| MCRNNN93S09C710W | terror   |
| MLVSTT52B71A794R | cdrom    |
| mzzgnn94m08a703j | qwerty   |
+------------------+----------+
\end{verbatim}

\medskip

\begin{verbatim}
Segretari:

+------------------+----------+
| CodiceFiscale    | Password |
+------------------+----------+
| mzqgrg94m08a703p | pass     |
+------------------+----------+

\end{verbatim}

\subsection{Sicurezza}

\textbf{Per motivi di sicurezza si è scelto di crittare, da php, tutte le password tramite hash crittografico sha-1.} 

\bigskip

\subsection{Dati utili}

Di seguito i \underline{codici fiscali} di tutti gli \textbf{allievi} e gli \textbf{insegnanti} per facilitare l'aggiunta di una lezione o di un saggio:

\medskip

\begin{parcolumns}{3}
   \colchunk{\begin{verbatim}
Allievi:
   
+------------------+
| CodiceFiscale    |
+------------------+
| RSSMRA92A01H501Y |
| zccgcm93d04g693a |
| zccgcq93d04g691a |
| zkcgcm93d04g692a |
+------------------+

   
   \end{verbatim}
             }
   \colchunk[2]{\begin{verbatim}
  Insegnanti:

  +------------------+
  | CodiceFiscale    |
  +------------------+
  | FSLFNC94S07G693X |
  | MCRNNN93S09C710W |
  | mzzgnn94m08a703j |
  | MLVSTT52B71A794R |
  +------------------+
   
   \end{verbatim}
   }
\end{parcolumns}







\end{document}
